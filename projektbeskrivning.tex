%%% Template originaly created by Karol Kozioł (mail@karol-koziol.net) and
%%% modified for ShareLaTeX use

\documentclass[a4paper,12pt]{article}

\usepackage[
  backend=biber,
  style=authoryear-ibid, % Motsvarar agsm.
  uniquename=init,
  giveninits,  % Ersätt hela förnamn med bara initialerna.
  maxnames=2,  % Ersätt med et al./m.fl. om det är fler än två förf.
  natbib=true,
  hyperref=true,
  % backref=true,
  doi=false,
  isbn=false,
  url=false
]{biblatex}
\addbibresource{exjobb.bib}
% Lite kortare referenslista utan URL:er och DOI:er.

\usepackage{csquotes} % Must be before babel.
\usepackage[swedish]{babel}
\usepackage[T1]{fontenc}
\usepackage[utf8]{inputenc}
\usepackage{pdfsync}
\synctex=1
\usepackage{parskip} % Package to tweak paragraph skipping.
\usepackage{hyperref}
\usepackage{xcolor}

\usepackage{fancyhdr}

\renewcommand\familydefault{\sfdefault}
\usepackage{tgheros}
\usepackage[defaultmono]{droidmono}
\usepackage[framed,numbered]{matlab-prettifier}
\makeatletter
\newcommand\BeraMonottfamily{%
  \def\fvm@Scale{0.6}% scales the font down
  \fontfamily{fvm}\selectfont% selects the Bera Mono font
}
\makeatother

\usepackage{amsmath,amssymb,amsthm,textcomp}
\usepackage{enumerate}
\usepackage{multicol}
\usepackage{tikz}
\usepackage{caption,subcaption}

\usepackage{calc}
\newenvironment{altDescription}[1][\quad]
  {\begin{list}{}{
   \renewcommand\makelabel[1]{\hfil\textsf{##1}}
   \settowidth\labelwidth{\makelabel{#1}}
   \setlength\leftmargin{\labelwidth+\labelsep}}}
  {\end{list}}
\usepackage{titling}

\title{Projektbeskrivning: identifiering av träffögonblick vid
  skottrampsträning inom ishockey med hjälp av datorseende}
  \author{Jonas Nockert, nockert@kth.se
  \\[0.5cm]{\small \textit{Handledare:} Björn Thuresson,
  \textit{Examinator:} Lars Arvestad}}
\renewcommand{\maketitlehookb}{\centering Examensarbete inom datalogi,
grundnivå, VT18}
\date{Maj 2018}

\begin{document}

\maketitle

\section*{Bakgrund}
För att bli en bra målskytt inom ishockey uppmuntrar Svenska Ishockeyförbundet
(\citeyear{Swehockey:2016}) spelare att skjuta minst 100 skott om dagen. Tiden
på isträningen räcker inte till detta och det är numera vanligt att klubbar
och spelare på alla nivåer bygger skottramper (se figur~\ref{fig:skottramp})
och använder dem för daglig träning.

\begin{figure}[ht]
  \centering
  \includegraphics[width=\linewidth]{photos/the-incredible-shooting-ramp-my-mom-built-for-me.png}
  \caption{Exempel på skottrampsträning utomhus
  (\href{https://yalewomenshockey.wordpress.com/2013/07/18/ywih-summer-blog-hanna-astrom/}{YALE Womens Hockey's Blog}).
  \label{fig:skottramp}}
\end{figure}

Idrottsforskaren Johnny Nilsson har tillsammans med kollegor på Dalarna
University filmat skottrampsträningar (figur~\ref{fig:leksand}) och manuellt
analyserat resultatet för att få grundläggande statistiska mått vad gäller
träffbilder. Denna metod har visat sig mycket tidskrävande, till stor del
på grund av att hög precision är betydligt svårare att uppnå än vad det först
verkar, något som kan ses i figur~\ref{fig:skott}.

\begin{figure}[ht]
  \centering
  \includegraphics[width=\linewidth]{photos/skottrampstraning-leksands-if.png}
  \caption{Bildruta från videoinspelning av skottrampsträning. Duken är
    en prototyp, anpassad för manuell analys. Kamerans placering bakom
    spelaren är inte optimal för automatisk analys då spelaren och spelarens
    klubba ibland skymmer träffpunkten. För automatisk analys är det bättre
    att placera kameran på sidan framför spelaren, i en vinkel mot målet.
  \label{fig:leksand}}
\end{figure}

\begin{figure}[ht]
  \centering
  \begin{subfigure}[t]{0.24\textwidth}
    \centering
    \includegraphics[width=\linewidth]{photos/skott1.png}
  \end{subfigure}%
  \hspace*{\fill}
  \begin{subfigure}[t]{0.24\textwidth}
    \centering
    \includegraphics[width=\linewidth]{photos/skott2.png}
  \end{subfigure}%
  \hspace*{\fill}
  \begin{subfigure}[t]{0.24\textwidth}
    \centering
    \includegraphics[width=\linewidth]{photos/skott3.png}
  \end{subfigure}%
  \hspace*{\fill}
  \begin{subfigure}[t]{0.24\textwidth}
    \centering
    \includegraphics[width=\linewidth]{photos/skott4.png}
  \end{subfigure}%

  \caption{Även manuellt kan det vara svårt att avgöra exakt var pucken
    träffar. Det gäller att försöka uppskatta avståndet i centimeter till
    den målpunkt som spelaren försökt träffa, här nummer 3. Rutnätet,
    $10 \times 10$~cm är till för okulär kvantifiering av träffdata (utsnitt
    ur träningsserie, filmad med videokamera i 60 fps).\label{fig:skott}}
\end{figure}

Nilssons problem gav 2017 upphov till ett MVK-projekt på KTH och en
prototyp som visade att det bör vara möjligt att automatisera processen med
precision som närmar sig motsvarande manuell analys om filmer i hög
upplösning och hög frekvens används som källmaterial.

Ett av grundproblemen vad gäller att med precision avgöra en pucks träffpunkt
är att position och riktning måste bestämmas i tre dimensioner. Med ett
tänkt plan parallellt med målburens framsida, längs mållinjen, så är det
endast när en puck passerar planet (i rätt riktning\footnote{Puckar kan t.ex.
rulla ut eller träffa andra puckar inne i målburen och få dem att studsa
ut.}) som dess koordinat ska registreras. Om registrering sker framför eller
bakom planet så påverkas precisionen negativt.

\begin{figure}[ht]
  \centering
  \includegraphics[width=\linewidth]{photos/3d-problem-two.jpg}
  \caption{I två dimensioner är det svårt att bestämma var pucken korsar
    planet längs mållinjen och målburens framsida.
  \label{fig:3d-problem2}}
\end{figure}

En vanlig kamera fångar bara två dimensioner och även om det ibland är möjligt
att återskapa tre dimensioner hos vissa objekt genom deras geometriska
egenskaper så låter sig detta inte göras med en hockeypuck som rör sig 30~m/s.
Med andra ord -- även om det skulle forskas fram en perfekt algoritm för att
följa en hockeypuck i två dimensioner så går det ändå inte att avgöra
\textit{exakt} när pucken passerar mållinjen\footnote{Givet att inte kameran
är riktad längs mållinjen men då går det inte längre att avgöra puckens
koordinat relativt målburens hörn.} (se figur~\ref{fig:3d-problem2}).

\begin{figure}[ht]
  \centering
  \includegraphics[width=\linewidth]{photos/shooting-tarp.jpg}
  \caption{Exempel på duk
  (\href{http://www.hockeyshot.se/HockeyShot-Extreme-Shooting-Tarp-p/target-tarp-032.htm}{hockeyshot.se}).
  \label{fig:duk}}
\end{figure}

%% \begin{figure}[ht]
%%   \centering
%%   \includegraphics[width=\linewidth]{photos/3d-problem.jpg}
%%   \caption{Träffpunkten ska enbart registreras första gången pucken
%%     passerar mållinjen efter ett skott. Studsande och rullande puckar ska
%%     inte registreras.
%%   \label{fig:3d-problem}}
%% \end{figure}

Nilssons lösning är att låta pucken träffa en tålig duk som motsvarar planet
parallellt med mållinjen. En pragmatisk och kostnadseffektiv metod, väl
anpassad till att skottrampsträning ofta redan genomförs med hjälp av duk
(se figur~\ref{fig:skottramp} och~\ref{fig:duk}). Mycket övergripande skulle
analysen då kunna brytas ned i tre steg:

\begin{enumerate}
  \item \label{enum:step1} Hitta dukens koordinatsystem givet godtycklig
    kameravinkel.
  \item \label{enum:step2} Hitta bildrutan då pucken träffar duken.
  \item \label{enum:step3} Hitta puckens koordinater i denna bildruta utifrån
    ovanstående koordinatsystem.
\end{enumerate}

Alla tre steg har en avgörande påverkan på analysens kvalitet och är
intressanta problem i sig själva. Det här arbetet kommer fokusera på
steg~\ref{enum:step2}, vilket verkar kunna formuleras som en variant av
rörelsedetektion: när duken börjar röra sig har den antagligen träffats av
en puck. Det finns dock ett antal komplicerande faktorer, t.ex.:

\begin{itemize}
  \item Allt som rör sig är inte pucken som träffar duken. Om inget annat så
    rör sig pucken framför duken innan den träffar. Utomhus kan vind få duken
    att bölja och skuggor från träd och moln kan röra sig över duken. Inomhus
    kan vibrationer i golvet orsaka skakningsoskärpa (vilket innebär rörelse
    i hela bilden).
  \item Om pucken träffar ett hörn rör sig duken väldigt lite jämfört med en
    träff i mitten. Utan någon form av kalibrering är inte spannet känt och
    det är inte uppenbart hur det går att avgöra om en rörelse är tillräcklig
    för att motsvara en träff, speciellt som skottet helt kan ha missat duken.
  \item Form och storlek på en puck är inte visuellt konsistent under rörelse.
    I vissa rotationer ser den ut som en cirkel, i andra rotationer som en
    ellips eller en tunn rektangel. Det är på så sätt svårt att definitivt
    avgöra om något är en puck eller inte.
  \item Det kan finnas fler än en puck i bild. Puckar blir framförallt
    liggande runt målet men de kan också glida eller rulla på marken en tid
    efter de träffat duken.
\end{itemize}

På grund av puckens form sker studsen mot duken inte kontrollerat och därför
är det viktigt att försöka identifiera en bildruta så nära tidpunkten då
pucken \textit{först} träffar duken. Det behöver dock inte vara en strikt
sekventiell sökning utan kan en annan, mer karaktäristisk, bildruta
tillförlitligt hittas så kan denna sedan användas som utgångspunkt för att
leta bakåt efter bildrutan av intresse.


\section*{Syfte}
Jämförelse (och optimering) av olika algoritmer för identifiering av tidpunkt
för träff (steg~\ref{enum:step2} ovan) av hockeypuck mot målduk. Kan
tidsstämplingen göras mer tillförlitlig under verkliga omständigheter bör
kvaliteten på den automatiska analysen öka avsevärt. Framför allt finns en
önskan om att minimera antalet identifikationer med stor tidsavvikelse, något
som kan leda till att träffen placeras långt (meter snarare än centimeter)
ifrån den faktiska träffpunkten.

Målet är inte en perfekt algoritm utan en algoritm med kända prestanda
gentemot noggrann manuell analys, samt statistisk referensdata så att
förbättrade algoritmer kan jämföras med resultaten från den här studien i
framtiden.

Syftet, i förlängningen, är att möjliggöra forskning inom ishockeyträning
baserad på stora mängder träningsdata över tid samt bädda för framtida
produkter för systematisk precisionsträning med hjälp av skottramp.


\section*{Metod och teknik}
Arbetet är av utforskande karaktär och är tänkt att utmynna i en algoritm
med kvantifierad prestanda. Arbetsmomenten består av datainsamling såväl
som teoretiska och praktiska undersökningar:

\begin{itemize}
  \item skapa referensdata genom att spela in filmer från skottrampsträning
    och noggrant manuellt analysera dem vad gäller tidpunkter för träff,
  \item hjälpa till att ta fram en duk som underlättar automatisk analys av
    skottrampsträning,
  \item med hjälp av Python och OpenCV undersöka bakgrundssubtraktion, den
    viktigaste klassen av metoder inom rörelsedetektion med fixerad kamera.
    Genom att lära upp en modell av bakgrunden i bilder kan förgrunden
    (intressanta förändringar) i nya bildrutor identifieras. Regn, löv som
    rör sig, ändrade ljusförhållanden och blinkande trafikljus är fenomen som
    t.ex.\ kan inkorporeras i bakgrundsmodellen. Modellen måste sedan
    kontinuerligt uppdateras för att hantera gradvis förändring över tid.

    Eftersom problemet är unikt, puckar som rör sig i bild måste t.ex.\
    ignoreras, är målet med undersökningen att hitta lovande specifika
    algoritmer som kan valideras praktiskt genom applikation på referensfilmer
    och jämföras med tidpunkter från manuell analys med hög temporal och
    spatial upplösning (''gold standard''),
  \item sätta samman bildbehandlings- och bildanalysmetoder till en algoritm
    vars tillförlitlighet för identifiering av träfftidpunkt är statistiskt
    kvantifierad och som kan visualiseras i ett övergripande flödesschema.
\end{itemize}


\section*{Avgränsning}
Eftersom rörelsedetektion ingår i ett stort och aktivt forskningsfält finns
ingen möjlighet att praktiskt undersöka det i helhet. Det är inom ramarna
för detta arbete inte rimligt att hysa någon förhoppning om att skapa en
perfekt algoritm. Därför bör arbetet kunna begränsas till framförallt metoder
och algoritmer som redan finns implementerade i OpenCV med bibehållet syfte.
Alla algoritmer i OpenCV har sitt ursprung i vetenskapliga tidskrifter.

Arbetet utförs med antagandet att steg~\ref{enum:step1} hanteras så nära
perfekt som möjligt och att steg~\ref{enum:step3} är relativt enkelt givet
att bildrutan för träff är identifierad. Det antas med andra ord att
steg~\ref{enum:step3} enbart använder tidpunkt/bildruta som indata från
steg~\ref{enum:step2} och att puckens position här bara är relevant om det
hjälper till att bestämma tidpunkt för träff.


\pagebreak
\section*{Planering/tidplan}
Kursen kommer läsas i halvfart över VT och sommar 18 med start i
kursen i vetenskaplighet som ingår i uppsatsarbetet. Uppsatsen skrivs på
engelska.

\subsection*{April}
\begin{itemize}
  \item Kurs i vetenskaplighet, 1.5 hp (tentamen 26 april). Klar.
\end{itemize}

\subsection*{Maj}
\begin{itemize}
  \item Övergripande undersöka domänen för rörelsedetektion. Teori, metoder,
    problem och möjligheter i relation till ovanstående. Samla relevanta
    källor.
  \item Undersöka domänen kring skottrampsträning för att i sin tur, i
    samarbete med Johnny Nilsson, ta fram en målduk och metod för mätning
    som väger in den algoritmiska komplexiteten på en datorseendelösning.
  \item Använda mobilkamera för att spela in preliminära filmer för
    att få testdata samt utvärdera utformningen på målduken.
    Mobil är den tänkta enheten för framtida träningsprodukter eller
    forskningsverktyg.
\end{itemize}

\subsection*{Juni}
\begin{itemize}
  \item Skriva på uppsatsens bakgrund
  \item Skriva syfte och begränsningar.
  \item Analysera filmer manuellt och etablera en initial referensdatamängd.
  \item Undersöka angreppsättet från det tidigare MVK-projektet, dels med
    avseende på analysmetod och dels med avseende på kvalitativt mått för
    testning som väger in både koordinatförskjutning och tidsförskjutning
    gentemot referensdata.
  \item Utgå från grundläggande, generella metoder för rörelsedetektion,
    tillsammans med Python och OpenCV, för att få en bättre bild av vad som
    måste hanteras specifikt för domänen.
  \item Justera duk och mätmetod baserat på resultatet så här långt. Spela
    in nya filmer med hockeyspelare i Dalarna, etablera kompletterande
    referensdatamängd. Jämföra med de tidigare resultaten.
\end{itemize}

\subsection*{Juli}
\begin{itemize}
  \item Skriva på metoddel och avhandlingsdel.
  \item Låna duk av Nilsson för att kunna spela in filmer under naturliga
    förhållanden med syfte att etablera referensdata vid olika väder, olika
    tider på dygnet, etc.
  \item Fortsätta undersöka metoder för rörelsedetektion med hjälp av
    referensdata (inspelade filmerna från maj, juni, samt filmer från
    MVK-projekt). Speciellt i avseende på fall som uppkommer under
    naturliga förhållanden.
  \item Vid behov spela in nya filmer med hockeyspelare i Dalarna.
\end{itemize}

\subsection*{Augusti}
\begin{itemize}
  \item Vid behov spela in nya filmer med hockeyspelare i Dalarna.
  \item Sammanställa resultaten i form av en jämförelse med manuellt
    analyserade filmer, både vad gäller sammantagen avvikelse och vad
    gäller utliggare och extremfall.
  \item Skriva metoddel, avhandlingsdel, diskussionsdel och slutsatser.
\end{itemize}

\subsection*{September}
\begin{itemize}
  \item Skriva avslutning och abstract.
  \item Färdigställa uppsats (med slut i mitten av september).
\end{itemize}

\printbibliography{}
\end{document}
